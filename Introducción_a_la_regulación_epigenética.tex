% Options for packages loaded elsewhere
\PassOptionsToPackage{unicode}{hyperref}
\PassOptionsToPackage{hyphens}{url}
\PassOptionsToPackage{dvipsnames,svgnames,x11names}{xcolor}
%
\documentclass[
  letterpaper,
  DIV=11,
  numbers=noendperiod]{scrartcl}

\usepackage{amsmath,amssymb}
\usepackage{iftex}
\ifPDFTeX
  \usepackage[T1]{fontenc}
  \usepackage[utf8]{inputenc}
  \usepackage{textcomp} % provide euro and other symbols
\else % if luatex or xetex
  \usepackage{unicode-math}
  \defaultfontfeatures{Scale=MatchLowercase}
  \defaultfontfeatures[\rmfamily]{Ligatures=TeX,Scale=1}
\fi
\usepackage{lmodern}
\ifPDFTeX\else  
    % xetex/luatex font selection
\fi
% Use upquote if available, for straight quotes in verbatim environments
\IfFileExists{upquote.sty}{\usepackage{upquote}}{}
\IfFileExists{microtype.sty}{% use microtype if available
  \usepackage[]{microtype}
  \UseMicrotypeSet[protrusion]{basicmath} % disable protrusion for tt fonts
}{}
\makeatletter
\@ifundefined{KOMAClassName}{% if non-KOMA class
  \IfFileExists{parskip.sty}{%
    \usepackage{parskip}
  }{% else
    \setlength{\parindent}{0pt}
    \setlength{\parskip}{6pt plus 2pt minus 1pt}}
}{% if KOMA class
  \KOMAoptions{parskip=half}}
\makeatother
\usepackage{xcolor}
\usepackage{svg}
\setlength{\emergencystretch}{3em} % prevent overfull lines
\setcounter{secnumdepth}{-\maxdimen} % remove section numbering
% Make \paragraph and \subparagraph free-standing
\makeatletter
\ifx\paragraph\undefined\else
  \let\oldparagraph\paragraph
  \renewcommand{\paragraph}{
    \@ifstar
      \xxxParagraphStar
      \xxxParagraphNoStar
  }
  \newcommand{\xxxParagraphStar}[1]{\oldparagraph*{#1}\mbox{}}
  \newcommand{\xxxParagraphNoStar}[1]{\oldparagraph{#1}\mbox{}}
\fi
\ifx\subparagraph\undefined\else
  \let\oldsubparagraph\subparagraph
  \renewcommand{\subparagraph}{
    \@ifstar
      \xxxSubParagraphStar
      \xxxSubParagraphNoStar
  }
  \newcommand{\xxxSubParagraphStar}[1]{\oldsubparagraph*{#1}\mbox{}}
  \newcommand{\xxxSubParagraphNoStar}[1]{\oldsubparagraph{#1}\mbox{}}
\fi
\makeatother


\providecommand{\tightlist}{%
  \setlength{\itemsep}{0pt}\setlength{\parskip}{0pt}}\usepackage{longtable,booktabs,array}
\usepackage{calc} % for calculating minipage widths
% Correct order of tables after \paragraph or \subparagraph
\usepackage{etoolbox}
\makeatletter
\patchcmd\longtable{\par}{\if@noskipsec\mbox{}\fi\par}{}{}
\makeatother
% Allow footnotes in longtable head/foot
\IfFileExists{footnotehyper.sty}{\usepackage{footnotehyper}}{\usepackage{footnote}}
\makesavenoteenv{longtable}
\usepackage{graphicx}
\makeatletter
\def\maxwidth{\ifdim\Gin@nat@width>\linewidth\linewidth\else\Gin@nat@width\fi}
\def\maxheight{\ifdim\Gin@nat@height>\textheight\textheight\else\Gin@nat@height\fi}
\makeatother
% Scale images if necessary, so that they will not overflow the page
% margins by default, and it is still possible to overwrite the defaults
% using explicit options in \includegraphics[width, height, ...]{}
\setkeys{Gin}{width=\maxwidth,height=\maxheight,keepaspectratio}
% Set default figure placement to htbp
\makeatletter
\def\fps@figure{htbp}
\makeatother

\KOMAoption{captions}{tableheading}
\makeatletter
\@ifpackageloaded{caption}{}{\usepackage{caption}}
\AtBeginDocument{%
\ifdefined\contentsname
  \renewcommand*\contentsname{Tabla de contenidos}
\else
  \newcommand\contentsname{Tabla de contenidos}
\fi
\ifdefined\listfigurename
  \renewcommand*\listfigurename{Listado de Figuras}
\else
  \newcommand\listfigurename{Listado de Figuras}
\fi
\ifdefined\listtablename
  \renewcommand*\listtablename{Listado de Tablas}
\else
  \newcommand\listtablename{Listado de Tablas}
\fi
\ifdefined\figurename
  \renewcommand*\figurename{Figura}
\else
  \newcommand\figurename{Figura}
\fi
\ifdefined\tablename
  \renewcommand*\tablename{Tabla}
\else
  \newcommand\tablename{Tabla}
\fi
}
\@ifpackageloaded{float}{}{\usepackage{float}}
\floatstyle{ruled}
\@ifundefined{c@chapter}{\newfloat{codelisting}{h}{lop}}{\newfloat{codelisting}{h}{lop}[chapter]}
\floatname{codelisting}{Listado}
\newcommand*\listoflistings{\listof{codelisting}{Listado de Listados}}
\makeatother
\makeatletter
\makeatother
\makeatletter
\@ifpackageloaded{caption}{}{\usepackage{caption}}
\@ifpackageloaded{subcaption}{}{\usepackage{subcaption}}
\makeatother

\ifLuaTeX
\usepackage[bidi=basic]{babel}
\else
\usepackage[bidi=default]{babel}
\fi
\babelprovide[main,import]{spanish}
% get rid of language-specific shorthands (see #6817):
\let\LanguageShortHands\languageshorthands
\def\languageshorthands#1{}
\ifLuaTeX
  \usepackage{selnolig}  % disable illegal ligatures
\fi
\usepackage{bookmark}

\IfFileExists{xurl.sty}{\usepackage{xurl}}{} % add URL line breaks if available
\urlstyle{same} % disable monospaced font for URLs
\hypersetup{
  pdftitle={Introducción a la regulación epigenética},
  pdfauthor={Sergio Melgar},
  pdflang={es-MX},
  colorlinks=true,
  linkcolor={blue},
  filecolor={Maroon},
  citecolor={Blue},
  urlcolor={Blue},
  pdfcreator={LaTeX via pandoc}}


\title{Introducción a la regulación epigenética}
\usepackage{etoolbox}
\makeatletter
\providecommand{\subtitle}[1]{% add subtitle to \maketitle
  \apptocmd{\@title}{\par {\large #1 \par}}{}{}
}
\makeatother
\subtitle{Basado en: Paro, P. D. R., Grossniklaus, P. D. U., Santoro, D.
R., \& Wutz, P. D. A. (2021). Biology of Chromatin. En Introduction to
Epigenetics {[}Internet{]}. Springer.
https://doi.org/10.1007/978-3-030-68670-3\_1}
\author{Sergio Melgar}
\date{}

\begin{document}
\maketitle


\subsection{Introducción a la Regulación
Epigenética}\label{introducciuxf3n-a-la-regulaciuxf3n-epigenuxe9tica}

\begin{itemize}
\item
  Todos los organismos heredan rasgos codificados en el ADN.
\item
  La expresión génica es una función clave del genoma.
\item
  Otras funciones son la replicación y la herencia.
\end{itemize}

\begin{figure}[H]

{\centering \includegraphics[width=5.20833in,height=\textheight]{Introducción_a_la_regulación_epigenética_files/mediabag/Epigenetic_mechanism.png}

}

\caption{Mecanismos epigenéticos.}

\end{figure}%

\subsection{Expresión génica}\label{expresiuxf3n-guxe9nica}

\begin{itemize}
\tightlist
\item
  Transformación de información genética a proteínas
\end{itemize}

\begin{figure}[H]

{\centering \includegraphics{Introducción_a_la_regulación_epigenética_files/mediabag/Gene_expression_euka.png}

}

\caption{Expresión de genes eucariotas}

\end{figure}%

\subsection{Organización del genoma}\label{organizaciuxf3n-del-genoma}

\begin{itemize}
\item
  El genoma se organiza en una fibra de cromatina con ADN.
\item
  En microscopio electrónico, la cromatina contiene partículas esféricas
  denominadas nucleosomas.
\end{itemize}

\begin{figure}[H]

{\centering \includegraphics{Introducción_a_la_regulación_epigenética_files/mediabag/ORNL_History_-446706.jpg}

}

\caption{Micrografía electrónica de la cromatina.}

\end{figure}%%
\begin{figure}[H]

{\centering \includegraphics{Introducción_a_la_regulación_epigenética_files/mediabag/Chromatin_and_histon.jpg}

}

\caption{La cromatina tiene ADN y proteínas.}

\end{figure}%

\subsection{Regulación de la expresión
génica}\label{regulaciuxf3n-de-la-expresiuxf3n-guxe9nica}

\begin{itemize}
\item
  La cromatina apoya la transducción de información epigenética en
  procesos regulatorios.
\item
  La cromatina es dinámica.
\item
  Modificaciones de histonas y la accesibilidad del ADN contribuyen a la
  regulación epigenética.
\end{itemize}

\begin{figure}[H]

{\centering \includegraphics[width=5.20833in,height=\textheight]{Introducción_a_la_regulación_epigenética_files/mediabag/Closed_and_relaxed_c.jpg}

}

\caption{Acetilación de histonas.}

\end{figure}%

\subsection{Descubrimiento del
Nucleosoma}\label{descubrimiento-del-nucleosoma}

\begin{itemize}
\item
  La microscopía electrónica reveló arreglos lineales de partículas
  esféricas.
\item
  Estas partículas están espaciadas regularmente y se han encontrado en
  muchos eucariotas.
\end{itemize}

\begin{figure}[H]

{\centering \includegraphics[width=5.20833in,height=\textheight]{Introducción_a_la_regulación_epigenética_files/mediabag/Chromatin_nucleofila.png}

}

\caption{Cromatina.}

\end{figure}%

\subsection{Descubrimiento del
Nucleosoma}\label{descubrimiento-del-nucleosoma-1}

\begin{itemize}
\item
  Los experimentos de digestión con nucleasas mostraron fragmentos de
  ADN a intervalos regulares de \textasciitilde150 pares de bases.
\item
  Estos hallazgos llevaron a la comprensión de la estructura
  nucleosómica.
\end{itemize}

\begin{figure}[H]

{\centering \includegraphics[width=5.20833in,height=\textheight]{Introducción_a_la_regulación_epigenética_files/mediabag/Chromatin_nucleofila.png}

}

\caption{Cromatina.}

\end{figure}%%
\begin{figure}[H]

{\centering \includegraphics{Introducción_a_la_regulación_epigenética_files/mediabag/Apoptotic_DNA_Ladder.png}

}

\caption{Escalera de ADN formada durante la apoptosis}

\end{figure}%

\subsection{La Estructura del
Nucleosoma}\label{la-estructura-del-nucleosoma}

\begin{itemize}
\item
  Un nucleosoma es la unidad básica repetitiva de la cromatina.
\item
  Consiste en 8 proteínas histonas y 146 pares de bases de ADN enrollado
  alrededor de ellas.
\item
  El octámero de histonas se compone de un par de cada una de las
  histonas H2A, H2B, H3 y H4.
\end{itemize}

\begin{figure}[H]

{\centering \includegraphics[width=5.20833in,height=\textheight]{Introducción_a_la_regulación_epigenética_files/mediabag/Nucleosome_structure.jpg}

}

\caption{Nucleosoma.}

\end{figure}%

\subsection{La Estructura del
Nucleosoma}\label{la-estructura-del-nucleosoma-1}

\begin{itemize}
\item
  Las histonas tienen superficies cargadas positivamente que interactúan
  con el ADN.
\item
  Tienen un dominio globular con hélice alfa.
\item
  Tienen colas N-terminales flexibles accesibles para modificaciones
  postraduccionales.
\end{itemize}

\begin{figure}[H]

{\centering \includegraphics{Introducción_a_la_regulación_epigenética_files/mediabag/Nucleosome_1KX5_colo.png}

}

\caption{Estructura del nucleosoma.}

\end{figure}%%
\begin{figure}[H]

{\centering \includegraphics[width=5.20833in,height=\textheight]{Introducción_a_la_regulación_epigenética_files/mediabag/Histone_tails_set_fo.jpg}

}

\caption{Nucleosoma modificado para activación.}

\end{figure}%

\subsection{Variantes de Histonas}\label{variantes-de-histonas}

\begin{itemize}
\tightlist
\item
  La composición de las histonas puede variar, introduciendo diferentes
  funcionalidades.
\item
  Las variantes de las histonas H3 y H2A son comunes, mientras que H2B y
  H4 son predominantemente invariantes.
\end{itemize}

\begin{figure}[H]

{\centering \includesvg[width=0.3\textwidth,height=\textheight]{Introducción_a_la_regulación_epigenética_files/mediabag/Basic_units_of_chrom.svg}

}

\caption{Composición de los nucleosomas.}

\end{figure}%

\subsection{Variantes de Histonas}\label{variantes-de-histonas-1}

\begin{itemize}
\tightlist
\item
  H3.1 y H3.2 son variantes que se depositan durante la replicación del
  ADN;
\item
  H3.3 está involucrada en el intercambio de histonas en sitios
  transcritos, heterocromatina pericéntrica y telomérica.
\item
  CENP-A reemplaza a la histona H3 en el centrómero, contribuyendo a su
  rigidez y su función del cinetocoro.
\end{itemize}

\begin{figure}[H]

{\centering \includesvg[width=1\textwidth,height=\textheight]{Introducción_a_la_regulación_epigenética_files/mediabag/Steps_in_nucleosome_.svg}

}

\caption{Ensamblaje de nucleosomas.}

\end{figure}%

\subsection{Variantes de Histonas}\label{variantes-de-histonas-2}

\begin{itemize}
\tightlist
\item
  Las histonas están asociadas con chaperonas o cargadoras (loaders)
  cuando no están en un contexto de cromatina. Por ejemplo:

  \begin{itemize}
  \tightlist
  \item
    NAP-1: Proteína ensambladora del nucleosoma 1 para H2A y H2B.
  \item
    CAF-1: Factor de ensamblaje de cromatina 1 para H3 y H4.
  \end{itemize}
\item
  La variante H2A.z está restringida a los promotores de genes activos.
\end{itemize}

\begin{figure}[H]

{\centering \includesvg[width=1\textwidth,height=\textheight]{Introducción_a_la_regulación_epigenética_files/mediabag/Steps_in_nucleosome_.svg}

}

\caption{Ensamblaje de nucleosomas.}

\end{figure}%

\subsection{Modificaciones de
Histonas}\label{modificaciones-de-histonas}

\begin{itemize}
\tightlist
\item
  Las modificaciones postraduccionales de las histonas y las
  modificaciones de las bases de ADN permiten la regulación epigenética.
\item
  Las enzimas que catalizan las modificaciones de las histonas se llaman
  escritoras.
\end{itemize}

\begin{figure}[H]

{\centering \includegraphics[width=1\textwidth,height=\textheight]{Introducción_a_la_regulación_epigenética_files/mediabag/Histone_modification.png}

}

\caption{Modificaciones de las histonas.}

\end{figure}%

\subsection{Modificaciones de
Histonas}\label{modificaciones-de-histonas-1}

\begin{itemize}
\tightlist
\item
  Las borradoras eliminan estas modificaciones, y las lectoras se unen a
  estas modificaciones y las interpretan.
\item
  Las modificaciones de las histonas pueden detectarse mediante
  anticuerpos utilizando métodos bioquímicos o técnicas de tinción.
\end{itemize}

\begin{figure}[H]

{\centering \includegraphics[width=1\textwidth,height=\textheight]{Introducción_a_la_regulación_epigenética_files/mediabag/Histone_modification.png}

}

\caption{Modificaciones de las histonas.}

\end{figure}%%
\begin{figure}[H]

{\centering \includegraphics{Introducción_a_la_regulación_epigenética_files/mediabag/NucleosomeKG.jpg}

}

\caption{Enzimas modificadoras de histonas.}

\end{figure}%

\subsection{Modificaciones de
Histonas}\label{modificaciones-de-histonas-2}

\begin{itemize}
\tightlist
\item
  Las técnicas de microscopía y las sondas fluorescentes pueden usarse
  para analizar los estados de modificación y el contexto genómico.
\end{itemize}

\begin{figure}[H]

{\centering \includegraphics{Introducción_a_la_regulación_epigenética_files/mediabag/lossy-page1-649px-Th.jpg}

}

\caption{Cromatina teñida de rojo por anticuerpos de proteína 1a
(heterocromatina) y en verde anticuerpos de contra H2Ax (eucromatina).}

\end{figure}%%
\begin{figure}[H]

{\centering \includegraphics[width=1\textwidth,height=\textheight]{Introducción_a_la_regulación_epigenética_files/mediabag/Relative_TERRA_Expre.jpg}

}

\caption{Modificaciones de las histonas.}

\end{figure}%

\subsection{Nomenclatura de las modificaciones de las
histonas}\label{nomenclatura-de-las-modificaciones-de-las-histonas}

H3K27me2:

\begin{itemize}
\item
  Modificación de H3
\item
  K: lisina
\item
  27: posición de la lisina
\item
  me2: dimetilación en la posición de la lisina.
\end{itemize}

\begin{figure}[H]

{\centering \includegraphics{Introducción_a_la_regulación_epigenética_files/mediabag/Representation_of_ch.jpg}

}

\caption{Marcas en cromatinas durante especialización de líneas
celulares.}

\end{figure}%%
\begin{figure}[H]

{\centering \includegraphics[width=1\textwidth,height=\textheight]{Introducción_a_la_regulación_epigenética_files/mediabag/Histone_modification.png}

}

\caption{Modificaciones de las histonas.}

\end{figure}%

\subsection{Modificaciones Combinatorias en la Heterocromatina
Pericéntrica}\label{modificaciones-combinatorias-en-la-heterocromatina-pericuxe9ntrica}

\begin{itemize}
\tightlist
\item
  H3K9me3 es establecida por metiltransferasas de histonas como Suv3-9h1
  y Suv3-9h2.
\item
  HP1 (proteína de heterocromatina 1) se une a H3K9me3 y es reclutada en
  la heterocromatina pericéntrica.
\end{itemize}

\begin{figure}[H]

{\centering \includegraphics{Introducción_a_la_regulación_epigenética_files/mediabag/Heterochromatin_stru.jpg}

}

\caption{Células madre embrionarias no diferenciadas (izquierda), y
células progenitoras neuronales (derecha). anticuerpos anti-HP1(verdes)
para heterocromatina, anticuerpos anti-H3K9me3 (rojo) and DAPI (azul).}

\end{figure}%%
\begin{figure}[H]

{\centering \includegraphics[width=5.20833in,height=\textheight]{Introducción_a_la_regulación_epigenética_files/mediabag/Karyotype_of_garlic_.png}

}

\caption{Centrómeros de una célula de ajo (rojo) teñidos con anticuerpo
contra la histona centromérica H3 propia de plantas (CENH3).}

\end{figure}%

\subsection{Identificación en alta resolución de las modificaciones de
las
histonas}\label{identificaciuxf3n-en-alta-resoluciuxf3n-de-las-modificaciones-de-las-histonas}

\begin{itemize}
\tightlist
\item
  La inmunoprecipitación de cromatina (ChIP) vincula histonas
  modificadas con secuencias de ADN.
\item
  ChIP utiliza anticuerpos para enriquecer una modificación específica
  de histona.
\item
  El ADN puede analizarse mediante varios métodos, incluyendo ChIP-seq,
  ChIP-array y PCR.
\item
  Esto permite obtener mapas genómicos de modificaciones de histonas a
  resolución nucleosómica.
\end{itemize}

\begin{figure}[H]

{\centering \includegraphics[width=1\textwidth,height=\textheight]{Introducción_a_la_regulación_epigenética_files/mediabag/Chip-seq.png}

}

\caption{Pasos básicos de la técnica del ChIP-seq.}

\end{figure}%

\subsection{Identificación en alta resolución de las modificaciones de
las
histonas}\label{identificaciuxf3n-en-alta-resoluciuxf3n-de-las-modificaciones-de-las-histonas-1}

\begin{figure}[H]

{\centering \includegraphics[width=1\textwidth,height=\textheight]{Introducción_a_la_regulación_epigenética_files/mediabag/ChIP_fig_4.png}

}

\caption{Pasos básicos de la tècnica del ChIP-array.}

\end{figure}%

\subsection{Escritoras, Lectoras y Borradoras de Modificaciones de
Histonas}\label{escritoras-lectoras-y-borradoras-de-modificaciones-de-histonas}

\begin{itemize}
\tightlist
\item
  Histona acetiltransferasas (HATs), histona metiltransferasas (HMTs) e
  histona quinasas utilizan cofactores de vías metabólicas para
  modificar histonas.
\item
  Las proteínas que contienen bromodominios se unen a lisinas acetiladas
  y atraen a otros reguladores y son consideradas lectoras.
\end{itemize}

\begin{figure}[H]

{\centering \includegraphics[width=0.25\textwidth,height=\textheight]{Introducción_a_la_regulación_epigenética_files/mediabag/1e6i_bromodomain.png}

}

\caption{Bromodominio de una proteína de levadura de cerveza (GCN5)}

\end{figure}%

\subsection{Escritores, Lectores y Borradores de Modificaciones de
Histonas}\label{escritores-lectores-y-borradores-de-modificaciones-de-histonas}

\begin{itemize}
\tightlist
\item
  La acetilación de histonas se cree que elimina cargas positivas de
  lisinas y reduce su afinidad al ADN.
\item
  Las acetilaciones están asociadas a promotores activos.
\item
  HP1 (Proteína de unión a la heterocromatina) se asocia con H3K9me3 a
  través de un cromodominio y es una lectora.
\item
  Los cromodominios se encuentran en proteínas asociadas a
  heterocromatina.
\end{itemize}

\begin{figure}[H]

{\centering \includegraphics[width=1\textwidth,height=\textheight]{Introducción_a_la_regulación_epigenética_files/mediabag/PDB_1pfb_EBI.jpg}

}

\caption{Cromodominio.}

\end{figure}%

\begin{figure}[H]

{\centering \includegraphics[width=5.20833in,height=\textheight]{Introducción_a_la_regulación_epigenética_files/mediabag/Heterochromatin_stru.jpg}

}

\caption{Células no diferenciadas ES (ESC, izquierda) derivadas de un
progenitor neuronal (NPC, derecha) teñidas con anticuerpos anti-HP1
(verde), anticuerpos anti-H3K9me3 (rojo) y DAPI (azul). En células ES,
la heterocromatina aparece mayor y conmenores focos mientras que en las
células progenitoras neuronales los focos son menores, más condensados y
numerosos. Barra = 10 μm.}

\end{figure}%

\subsection{Escritores, Lectores y Borradores de Modificaciones de
Histonas}\label{escritores-lectores-y-borradores-de-modificaciones-de-histonas-1}

\begin{itemize}
\tightlist
\item
  Los cromodominios reconocen resíduos de lisina metilados.
\item
  La proteína Polycomb (Pc) contiene un cromodominio específico para
  H3K27me3 y se encargan de silenciar genes
\item
  Las proteínas 14-3-3 tienen alta especificidad por la histona
  H3S10pK14ac doblemente modificada.
\item
  La señalización celular y las modificaciones combinatorias de histonas
  contribuyen a mecanismos complejos de regulación génica.
\end{itemize}

\begin{figure}[H]

{\centering \includegraphics{Introducción_a_la_regulación_epigenética_files/mediabag/Histone_methyltransf.jpg}

}

\caption{Metiltransferasas y demetilasas de histonas funcionan de manera
antagonista. Pueden permitir diferenciación de células madre
embrionarias.}

\end{figure}%

\subsection{Modificaciones del ADN}\label{modificaciones-del-adn}

\begin{itemize}
\tightlist
\item
  Las modificaciones del ADN añaden información al genoma sin cambiar la
  secuencia de ADN.
\item
  Al sintetizar la nueva hebra, las modificaciones son heredadas.
\item
  Los mecanismos de herencia epigenética es mejor entendido en las
  modificaciones del ADN.
\end{itemize}

\begin{figure}[H]

{\centering \includegraphics[width=5.20833in,height=\textheight]{Introducción_a_la_regulación_epigenética_files/mediabag/Epigenetics_modifica.png}

}

\caption{Esta imagen ilustra posibles modificaciones epigenéticas en el
ADN y en las histonas, como metilación de los sitios CpG, acetilación y
metilación de las colas de las histonas.}

\end{figure}%

\subsection{Modificaciones del ADN}\label{modificaciones-del-adn-1}

\begin{itemize}
\tightlist
\item
  La metilación del ADN es un sistema más simple para la heredabilidad
  epigenética.
\item
  5mC se encuentra preferentemente en el contexto de dinucleótidos CG en
  genomas de mamíferos.
\end{itemize}

\begin{figure}[H]

{\centering \includegraphics{Introducción_a_la_regulación_epigenética_files/mediabag/lossless-page1-800px.png}

}

\caption{Metilación de la citosina.}

\end{figure}%%
\begin{figure}[H]

{\centering \includegraphics[width=5.20833in,height=\textheight]{Introducción_a_la_regulación_epigenética_files/mediabag/Epigenetics_modifica.png}

}

\caption{Esta imagen ilustra posibles modificaciones epigenéticas en el
ADN y en las histonas, como metilación de los sitios CpG, acetilación y
metilación de las colas de las histonas.}

\end{figure}%

\subsection{Modificaciones del ADN}\label{modificaciones-del-adn-2}

\begin{itemize}
\tightlist
\item
  Las enzimas DNMT (ADN metiltransferasas) de mantenimiento restauran
  los patrones de metilación en las cadenas de ADN recién sintetizadas.
\end{itemize}

\begin{figure}[H]

{\centering \includegraphics{Introducción_a_la_regulación_epigenética_files/mediabag/lossless-page1-800px.png}

}

\caption{Metilación de la citosina.}

\end{figure}%%
\begin{figure}[H]

{\centering \includegraphics[width=5.20833in,height=\textheight]{Introducción_a_la_regulación_epigenética_files/mediabag/Epigenetics_modifica.png}

}

\caption{Esta imagen ilustra posibles modificaciones epigenéticas en el
ADN y en las histonas, como metilación de los sitios CpG, acetilación y
metilación de las colas de las histonas.}

\end{figure}%

\begin{center}\rule{0.5\linewidth}{0.5pt}\end{center}

\subsection{Organización y Compartimentación de la
Cromatina}\label{organizaciuxf3n-y-compartimentaciuxf3n-de-la-cromatina}

\begin{itemize}
\tightlist
\item
  La cromatina mantiene una distribución relativamente estable en el
  núcleo.
\item
  Los cromosomas ocupan volúmenes discretos llamados territorios
  cromosómicos (CTs).
\end{itemize}

\begin{figure}[H]

{\centering \includesvg[width=1\textwidth,height=\textheight]{Introducción_a_la_regulación_epigenética_files/mediabag/Chromosome_territori.svg}

}

\caption{Territorios cromosómicos.}

\end{figure}%

\begin{figure}[H]

{\centering \includegraphics[width=1\textwidth,height=\textheight]{Introducción_a_la_regulación_epigenética_files/mediabag/MouseChromosomeTerri.jpg}

}

\caption{Territorios cromosómicos en ratón, cromosoma 2 (rojo) y 9
(verde) en fondo azul.}

\end{figure}%

\begin{center}\rule{0.5\linewidth}{0.5pt}\end{center}

\subsection{Organización y Compartimentación de la
Cromatina}\label{organizaciuxf3n-y-compartimentaciuxf3n-de-la-cromatina-1}

\begin{itemize}
\tightlist
\item
  Los nucléolos son estructuras prominentes donde se transcribe el ARN
  ribosómico.
\item
  Los compartimentos intercromatínicos (IC) son canales dentro de la
  cromatina que permiten el acceso a los genes.
\end{itemize}

\begin{figure}[H]

{\centering \includesvg{Introducción_a_la_regulación_epigenética_files/mediabag/CT-IC-Modell-DE.svg}

}

\caption{Modelo de territorio cromosómico y compartimento
intercromatínico (CT-IC).}

\end{figure}%%
\begin{figure}[H]

{\centering \includesvg[width=5.20833in,height=\textheight]{Introducción_a_la_regulación_epigenética_files/mediabag/Diagram_human_cell_n.svg}

}

\caption{Núcleo.}

\end{figure}%

\begin{center}\rule{0.5\linewidth}{0.5pt}\end{center}

\subsection{Organización y Compartimentación de la
Cromatina}\label{organizaciuxf3n-y-compartimentaciuxf3n-de-la-cromatina-2}

\begin{itemize}
\tightlist
\item
  La heterocromatina pericéntrica se organiza en dominios con
  modificaciones específicas de histonas .
\item
  Los dominios de asociación topológica (TADs) son regiones de cromatina
  con altas frecuencias de interacción.

  \begin{itemize}
  \tightlist
  \item
    Los ``compartimentos A'' se correlacionan con marcas de cromatina
    activa y los ``compartimentos B'' con cromatina inactiva.
  \end{itemize}
\end{itemize}

\begin{figure}[H]

{\centering \includegraphics[width=5.20833in,height=\textheight]{Introducción_a_la_regulación_epigenética_files/mediabag/Structural_organizat.png}

}

\caption{Organización estructural de la cromatina. (A) Territorios
cromosómicos. (B) Dominios de asociación topológica (TADs). (C) TAD con
interacciones entre elementos regulatorios y genes.}

\end{figure}%

\subsection{Organización y Compartimentación de la
Cromatina}\label{organizaciuxf3n-y-compartimentaciuxf3n-de-la-cromatina-3}

\begin{itemize}
\item
  Los TADs se determinan mediante técnicas que permiten detectar las
  secuencias de ADN que se asocian a otras por su proximidad.
\item
  En la Captura de Conformación de Cromatina (3C) se hacen ligamientos
  cruzados entre el ADN cercano y proteínas asociadas mediante
  formaldehído.
\item
  Luego se digiere con nucleasas.
\item
  Luego se secuencian los fragmentos de ADN que se encontraban cercanos.
\item
  Se grafica en una matriz.
\end{itemize}

\begin{figure}[H]

{\centering \includegraphics{Introducción_a_la_regulación_epigenética_files/mediabag/HiCschematic.png}

}

\caption{Procedimiento para Hi-C y detección de TADs.}

\end{figure}%

\subsection{Complejos de Mantenimiento Estructural de Cromosomas
(SMC)}\label{complejos-de-mantenimiento-estructural-de-cromosomas-smc}

\begin{itemize}
\tightlist
\item
  La formación de TADs requiere proteínas de la familia SMC.
\item
  Las proteínas SMC están involucradas en la condensación y segregación
  de los cromosomas durante la mitosis.
\item
  Los complejos cohesina y condensina son dos complejos SMC distintos.
\item
  Las proteínas SMC tienen dominios de cabeza ATPasa que hidrolizan ATP
  para ejercer función motora.
\end{itemize}

\begin{figure}[H]

{\centering \includesvg[width=5.20833in,height=\textheight]{Introducción_a_la_regulación_epigenética_files/mediabag/Models_of_SMC_and_co.svg}

}

\caption{Proteínas SMC.}

\end{figure}%

\begin{center}\rule{0.5\linewidth}{0.5pt}\end{center}

\subsection{Complejos de Mantenimiento Estructural de Cromosomas
(SMC)}\label{complejos-de-mantenimiento-estructural-de-cromosomas-smc-1}

\begin{itemize}
\tightlist
\item
  Las cohesinas median la cohesión de las cromátidas hermanas y también
  están involucradas en la regulación génica.
\item
  Los bucles de cromatina tienen en su base sitios CTCF (Factor de unión
  a un sitio CCCTC)
\item
  Dichos sitios están orientados hacia cada otro, lo que sugiere un
  bloqueo direccional del deslizamiento de la cohesina.
\end{itemize}

\begin{figure}[H]

{\centering \includegraphics[width=5.20833in,height=\textheight]{Introducción_a_la_regulación_epigenética_files/mediabag/Cohesin_during_repli.jpg}

}

\caption{Cohesinas durante la replicación.}

\end{figure}%

\subsection{Complejos de Mantenimiento Estructural de Cromosomas
(SMC)}\label{complejos-de-mantenimiento-estructural-de-cromosomas-smc-2}

\begin{itemize}
\item
  Las condensinas se encargan de la condensación de los cromosomas
  mitóticos.

  \begin{figure}[H]

  {\centering \includegraphics{Introducción_a_la_regulación_epigenética_files/mediabag/Condensation1.png}

  }

  \caption{Célula en interfase (izquierda) y juego de cromosomas
  mitóticos (derecha).}

  \end{figure}%
\end{itemize}

\begin{figure}[H]

{\centering \includegraphics{Introducción_a_la_regulación_epigenética_files/mediabag/3condensins2-en-.png}

}

\caption{Composición de subunidades de complejos de condensina.}

\end{figure}%

\section{Gracias}\label{gracias}




\end{document}
